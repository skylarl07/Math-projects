\documentclass[12pt]{amsart}
\usepackage{amsmath, amsthm, graphicx, fullpage, bm, mathrsfs, mathabx, enumerate, marginnote, color, hyperref, wasysym}

%%%%%for footnote%%%%%
\usepackage[symbol]{footmisc}
\renewcommand{\thefootnote}{\fnsymbol{footnote}}
%%%%%%%%%%%%%%%%

%%%%%set of commands defined by me%%%%%
\newcommand{\abs}[1]{\ensuremath{\left\lvert #1\right\rvert}}
\newcommand*\diff{\mathop{}\!\mathrm{d}}
\newcommand{\NOT}[1]{\ensuremath{\sim\hspace{-0.05in}#1}}
\renewcommand*\divides{\hspace{0.05in}|\hspace{0.05in}}
\renewcommand*\div{\hspace{0.05in}\text{div}\hspace{0.05in}}
\renewcommand*\mod{\hspace{0.05in}\text{mod}\hspace{0.05in}}
\newcommand{\floor}[1]{\ensuremath{\left\lfloor #1\right\rfloor}}
\newcommand{\ceil}[1]{\ensuremath{\left\lceil #1\right\rceil}}
%%%%%%%%%%%%%%%%%%%%%%%%%%

%%%%%%%%%%%%%%%%%%%%%%%%%%

%%%%%%%%%%%%%%%%%%set notation%%%%%%%%%%%%%%%%%%
%https://tex.stackexchange.com/questions/253077/how-do-you-create-a-set-in-latex
\usepackage{mathtools}
\DeclarePairedDelimiterX\set[1]\lbrace\rbrace{\def\given{\;\delimsize\vert\;}#1}
%%%%%%%%%%%%%%%%%%%%%%%%%%%%%%%%%%%%%%%%%%

\begin{document}
%\pagenumbering{gobble}

\hfill\makebox[2.5in][l]{\textbf{Name: Skylar Liang}}

\vspace{0.3in}
\begin{center}
{\Large{\textbf{MATH 3034}}}

\vspace{0.1in}
{\large{Problem Set 4 (Due Monday, October 14, 2019)}}
\end{center}

\vspace{0.3in}
\noindent
\underline{Instructions}
\begin{itemize}
\vspace{0.1in}
\item You must print the problem set, write solutions on the printout (not on a separate sheet of paper), and staple all pages together.  Alternatively,  You will lose 1 point for not following this rule.
\vspace{0.1in}
\item Check the list of definitions and mathematical results you are allowed to use (available on Canvas).  You may not quote theorems found in a random book, including AAR.
\vspace{0.1in}
\item No late submission will be accepted.
\vspace{0.1in}
\item Each problem set will be graded out of 20 points based on your work on a few selected problems and the completeness of the rest of homework.
\vspace{0.1in}
\item You will receive 1 point for typing up your solutions using \LaTeX, as long as your problem set score does not exceed 20 points.  If you decide to do so, then download the corresponding tex file from Canvas and modify it (do not erase or change problem statements).  All VT students receive free Pro accounts on \href{https://www.overleaf.com/edu/vtech}{Overleaf}.
\vspace{0.1in}
\item You are encouraged to work with your classmates, but \textbf{you must write up your own solutions}.  Any form of plagiarism (e.g. copying your friend's tex file) will not be tolerated and will be reported to the Office of Academic Integrity.
\vspace{0.1in}
\item My solution will be posted within 24 hours of the deadline.  If there is a major mistake, the first student to point it out will receive 1 point.  I highly recommend that you make a copy of your problem set before you submit so that you can correct your mistakes immediately.
\end{itemize}

\vfill
\noindent
\textbf{Honor Pledge}: I did not give or receive any unauthorized assistance on this problem set.

\vspace{0.5in}
\hfill\makebox[3.0in]{\hrulefill}

\hfill\makebox[3.0in]{Signature}

\newpage
\noindent
\underline{Theorem} (proved later)

\vspace{0.05in}
\noindent
Let $n\in\mathbb{Z}$.  If $3\divides n^{2}$, then $3\divides n$.

\vspace{0.2in}
\begin{enumerate}[{\bfseries 1.}]
\item Consider the following statement.

	\vspace{0.1in}
	\begin{center}
	\noindent
	\textit{$\sqrt{3}$ is irrational}.
	\end{center}

	\begin{enumerate}[(a)]
	\vspace{0.1in}
	\item State the negation.
	\begin{normalize}
	\vspace{0.1in}
	\\ $\sqrt{3}$ is rational.
	\end{normalize}
	\item Prove the original statement using proof by contradiction.  Use the theorem above if necessary.
	\begin{normalize}
	\vspace{0.1in}
	\begin{proof}
	\\ Suppose to the contrary, that $\sqrt{3}$ is rational, then there exist $m, n \in\mathbb{Z}$ such that $\sqrt{3} = \frac{m}{n}$ and $n \neq 0$.
	\\ By dividing by any common factors if necessary, we may assume that $m$ and $n$ have no common factor.
	\\ Squiring both sides, we get $3 = \frac{m^2}{n^2}$, so $m^2 = 3\cdot n^2$.
	\\ Because $n^2 \in\mathbb{Z}$, this implies that  $m^2$ is divisible by $3$ and $m \in\mathbb{Z}$, so $m$ is divisible by $3$ by the theorem given.
	\\ Thus, there exits $k \in\mathbb{Z}$ such that $m = 3k$, and we obtain $3n^2 = (3k)^2 = 9k^2$, so $n^2 = 3k^2$, which again implies that $n^2$ is divisible by $3$ and $n\in\mathbb{Z}$, so $n$ is divisible by $3$ by theorem given.
	\\ Therefore $m$ and $n$ are both divisible by 3, which is a contradiction. 
	\end{proof}
	\end{normalize}
	\vfill
	\vfill
	\vfill
	\vfill
	\vfill
	\end{enumerate}

\newpage
\item Consider the following subsets of $\mathbb{R}$.

	\vspace{0.05in}
	\noindent
	$A=\set{n\in\mathbb{Z}^{+}\given n\text{ is composite and } n<7}$

	\vspace{0.05in}
	\noindent
	$B=\set{n\in\mathbb{Q} \given n^{2}=n}$

	\vspace{0.05in}
	\noindent
	$C=\set{n\in\mathbb{Z}\given n\text{ is a perfect square and } 0<n\leq 10}$

	\begin{enumerate}[(a)]
	\vspace{0.1in}
	\item What is $A$?
	\begin{normalize}
	\vspace{0.1in}
	\\ $A = \set{4, 6}$
	\end{normalize}
	\vspace{0.1in}
	\item What is $B$?
	\begin{normalize}
	\vspace{0.1in}
	\\ $B = \set{0, 1}$
	\end{normalize}
	\vspace{0.1in}
	\item What is $C$?
	\begin{normalize}
	\vspace{0.1in}
	\\ $C = \set{1, 4, 9}$
	\end{normalize}
	\vspace{0.1in}
	\item What is $A\cup C$?
	\begin{normalize}
	\vspace{0.1in}
	\\ $A\cup C = \set{0, 1, 4, 6}$
	\end{normalize}
	\vspace{0.1in}
	\item What is $B\cap C$?
	\begin{normalize}
	\vspace{0.1in}
	\\ $B\cap C = \set{1}$ 
	\end{normalize}
	\vspace{0.1in}
	\item What is $C-B$?
	\begin{normalize}
	\vspace{0.1in}
	\\ $C - B = \set{4, 9}$
	\end{normalize}
	\vspace{0.1in}
	\item What is $P(C)$?
	\begin{normalize}
	\vspace{0.1in}
	\\ $P(C) = \set{\emptyset, \set{1}, \set{4}, \set{9}, \set{1, 4}, \set{1, 9}, \set{4, 9}, \set{1, 4, 9}}$
	\end{normalize}
	\vspace{0.1in}
	\item What is $C\times A$?
	\begin{normalize}
	\vspace{0.1in}
	\\ $C \times A = \set{(1, 4), (1, 6), (4, 4), (4, 6), (9, 4), (9, 6)}$
	\end{normalize}
	\vspace{0.1in}
	\item Which sets are disjoint?
	\begin{normalize}
	\vspace{0.1in}
	\\ $C - B$ and $B$, $B \cap C$ and $A$
	\end{normalize}
	\end{enumerate}

\newpage
\item Consider the subsets
	$$A_{k}=[2k, 2(k+1)],\ B_{k}=[e^{-k}, e^{k}),\text{ and } C_{k}=(1-2^{-k}, k!]$$
	of $\mathbb{R}$ (where $k\in\mathbb{Z}^{+}$).
	Determine the given set.  You do not need to justify your answer.
	\begin{enumerate}[(a)]
	\vspace{0.1in}
	\item $\left(\displaystyle\bigcup_{i=1}^{4}A_{i}\right)'$
	    \begin{normalize}
    	\vspace{0.1in}
    	\\ $\displaystyle\bigcup_{i=1}^{4}A_{i} = A_{1} \cup A_{2} \cup A_{3} \cup A_{4} = [2, 4] \cup [4, 6] \cup [6, 8] \cup [8, 10] = [2, 10]$
    	\\ $\left(\displaystyle\bigcup_{i=1}^{4}A_{i}\right)' = (-\infty, 2) \cup (10, \infty)$
    	\end{normalize}
	\vspace{0.1in}
	\item $\displaystyle\bigcup_{j=1}^{2}\left(A_{2j}\cap A_{2j+1}\right)$
		\begin{normalize}
    	\vspace{0.1in}
    	\\ $= (A_{2} \cap A_{3}) \cup (A_{4} \cap A_{5}) = ([4, 6] \cap [6, 8]) \cup ([8, 10] \cap [10, 12])$
    	\\ $= \set{6} \cup \set{10} = \set{6, 10}$
    	\end{normalize}
	\vspace{0.1in}
	\item $\displaystyle\bigcup_{i=1}^{\infty}B_{i}$
		\begin{normalize}
    	\vspace{0.1in}
    	\\ $= B_{1} \cup B_{2} \cup B_{3} \cup ... = [e^{-1}, e) \cup [e^{-2}, e^2) \cup [e^{-3}, e^3) \cup ... = (0, \infty)$
    	\end{normalize}
	\vspace{0.1in}
	\item $\displaystyle\bigcap_{i=1}^{\infty}C_{i}$
		\begin{normalize}
    	\vspace{0.1in}
    	\\ $= C_{1} \cap C_{2} \cap C_{3} \cap ... = (\frac{1}{2}, 1] \cap (\frac{3}{4}, 2] \cap (\frac{7}{8}, 6] \cap ... = \emptyset$
    	\end{normalize}
	\end{enumerate}

\newpage
\item\begin{enumerate}[(a)]
	\item Let $A$ be the set of all prime numbers and $B$ be the set of all composite numbers.  \\
		Is $\{A, B\}$ a partition of $\mathbb{Z}^{+}$?  Explain.
		\begin{normalize}
    	\vspace{0.1in}
    	\\ $A \cup B = \mathbb{R} - \set{1} \neq \mathbb{R}$
    	\\ Not a partition
    	\end{normalize}
	\vspace{0.1in}
	\item Let $A=\set{n\in\mathbb{Z}\given n\text{ is divisible by }2\text{ and }3}$, $B=\set{n\in\mathbb{Z}\given n\text{ is not divisible by }2}$, and $C=\set{n\in\mathbb{Z}\given n\text{ is not divisible by }3}$.\\
		Is $\{A, B, C\}$ a partition of $\mathbb{Z}$?  Explain.
		\begin{normalize}
    	\vspace{0.1in}
    	\\ $A \cup B \cup C = \mathbb{R}$
    	\\ $A \cap B = \emptyset$, $A \cap C = \emptyset$, $B \cap C = \set{n \in\mathbb{Z}\given n \text{ is not divisible by }2 \text{ or } 3}$
    	\\ Not a partition
    	\end{normalize}
	\vspace{0.1in}
	\item Let $A=\set{x\in\mathbb{Q}\given x^{2}\leq 2}$ and $B=\set{x\in\mathbb{Q}\given x^{2}\geq 2}$.  \\
		Is $\{A, B\}$ a partition of $\mathbb{Q}$?  Explain.
		\begin{normalize}
    	\vspace{0.1in}
    	\\ $A \cup B = \mathbb{Q}$
    	\\ $A \cap B = \emptyset$
    	\\ Partition
    	\end{normalize}
	\vspace{0.1in}
	\item Let $A=\set{x\in\mathbb{R}\given x=\sqrt{y}\text{ for some }y\in\mathbb{R}}$ and $B=\set{x\in\mathbb{R}\given\ceil{x}<1}$.  \\
		Is $\{A, B\}$ a partition of $\mathbb{R}$?  Explain.
		\begin{normalize}
    	\vspace{0.1in}
    	\\ $A \cup B = \mathbb{R}$
    	\\ $A \cap B = \set{0}$
    	\\ Not a partition
    	\end{normalize}
	\end{enumerate}
\end{enumerate}

\newpage
\noindent
For the rest of this problem set, let $A$, $B$, $C$, and $D$ be subsets of a universal set $U$.

\vspace{0.2in}
\begin{enumerate}[{\bfseries 1.}]
\addtocounter{enumi}{4}
\item Prove the distributive law

	\vspace{0.1in}
	\begin{center}
	\noindent
	\textit{$A\cap (B\cup C)=(A\cap B)\cup (A\cap C)$.}
	\end{center}
\begin{normalize}
\begin{proof}
\vspace{0.1in}
\\ (1) We will prove $A\cap (B\cup C) \subseteq (A\cap B)\cup (A\cap C)$.
\\ Let $x \in A\cap (B\cup C)$, so $x \in A$ and $x\in B\cup C$.
\\ Case 1 ($x\in B$)
\\ Because $x \in A$ and $x \in B$, we have $x \in A\cap B$, so $x \in (A\cap B)\cup (A\cap C)$.
\\ Case 2 ($x \in C$)
\\ Because $x \in A$ and $x \in C$, we have $x \in A\cap C$, so $x \in (A\cap B)\cup (A\cap C)$.
\\ (2) We will prove $(A\cap B)\cup (A\cap C) \subseteq A\cap (B\cup C)$.
\\ Let $x \in (A\cap B)\cup (A\cap C)$.
\\ Case 1 ($x\in A\cap B$)
\\ Because $x\in A\cap B$, we have $x\in A$ and $x\in B$, so $x\in B\cup C$, hence $x\in A\cap (B\cup C)$.
\\ Case 2 ($x\in A\cap C$)
\\ Because $x\in A\cap C$, we have $x\in A$ and $x\in C$, so $x\in B\cup C$, hence $x\in A\cap (B\cup C)$.
\end{proof}
\end{normalize}

\newpage
\item Prove the following statement.

	\vspace{0.1in}
	\begin{center}
	\noindent
	\textit{$(A\times B)\cap (C\times D)=(A\cap C)\times (B\cap D)$}
	\end{center}
	
\begin{normalize}
\begin{proof}
\vspace{0.1in}
\\ (1) We will prove $(A\times B)\cap (C\times D) \subseteq (A\cap C)\times (B\cap D)$.
\\ Let $(x, y)\in (A\times B)\cap (C\times D)$, so $(x, y)\in A\times B$ and $(x, y)\in C\times D$. Then $x\in A$, $x\in C$, and $y\in B$, $y\in D$
\\ So we have $x\in A\cap C$, and $y\in B\cap D$, hence $(x, y)\in (A\cap C)\times (B\cap D)$.
\\ (2) We will prove $(A\cap C)\times (B\cap D) \subseteq (A\times B)\cap (C\times D)$.
\\ Let $(x, y)\in (A\cap C)\times (B\cap D)$, so $x\in A\cap C$, and $y\in B\cap D$. Then $x\in A$ and $x\in C$, $y\in B$ and $y\in D$. 
\\ So we have $(x, y)\in A\times B$, and $(x, y)\in C\times D$, hence $(x, y)\in (A\times B)\cap (C\times D)$.
\end{proof}
\end{normalize}
	\vfill
	\vfill
\vspace{0.2in}
\item Prove the following statement.

	\vspace{0.1in}
	\begin{center}
	\noindent
	\textit{$(A\times B)\cup (C\times D)\subseteq (A\cup C)\times (B\cup D)$}
	\end{center}

\begin{normalize}
\begin{proof}
\vspace{0.1in}
\\ Let $(x, y)\in (A\times B)\cup (C\times D)$.
\\ Case 1 ($(x, y)\in A\times B$)
\\ Because we have $(x, y)\in A\times B$, $x\in A$ and $y\in B$. So $x\in A\cup C$ and $y\in B\cup D$, hence $(x, y)\in (A\cup C)\times (B\cup D)$.
\\ Case 2 ($(x, y)\in C\times D$)
\\ Because we have $(x, y)\in C\times D$, $x\in C$ and $y\in D$. So $x\in A\cup C$ and $y\in B\cup D$, hence $(x, y)\in (A\cup C)\times (B\cup D)$.
\end{proof}
\end{normalize}
	\vfill

\end{enumerate}
\end{document}
