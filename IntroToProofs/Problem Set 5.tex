\documentclass[12pt]{amsart}
\usepackage{amsmath, amsthm, graphicx, fullpage, bm, mathrsfs, mathabx, enumerate, marginnote, color, hyperref, wasysym}

%%%%%for footnote%%%%%
\usepackage[symbol]{footmisc}
\renewcommand{\thefootnote}{\fnsymbol{footnote}}
%%%%%%%%%%%%%%%%

%%%%%set of commands defined by me%%%%%
\newcommand{\abs}[1]{\ensuremath{\left\lvert #1\right\rvert}}
\newcommand*\diff{\mathop{}\!\mathrm{d}}
\newcommand{\NOT}[1]{\ensuremath{\sim\hspace{-0.05in}#1}}
\renewcommand*\divides{\hspace{0.05in}|\hspace{0.05in}}
\renewcommand*\div{\hspace{0.05in}\text{div}\hspace{0.05in}}
\renewcommand*\mod{\hspace{0.05in}\text{mod}\hspace{0.05in}}
\newcommand{\floor}[1]{\ensuremath{\left\lfloor #1\right\rfloor}}
\newcommand{\ceil}[1]{\ensuremath{\left\lceil #1\right\rceil}}
%%%%%%%%%%%%%%%%%%%%%%%%%%

%%%%%%%%%%%%%%%%%%%%%%%%%%

%%%%%%%%%%%%%%%%%%set notation%%%%%%%%%%%%%%%%%%
%https://tex.stackexchange.com/questions/253077/how-do-you-create-a-set-in-latex
\usepackage{mathtools}
\DeclarePairedDelimiterX\set[1]\lbrace\rbrace{\def\given{\;\delimsize\vert\;}#1}
%%%%%%%%%%%%%%%%%%%%%%%%%%%%%%%%%%%%%%%%%%

\begin{document}
%\pagenumbering{gobble}

\hfill\makebox[2.5in][l]{\textbf{Name: Skylar Liang}}

\vspace{0.3in}
\begin{center}
{\Large{\textbf{MATH 3034}}}

\vspace{0.1in}
{\large{Problem Set 5 (Due Monday, October 21, 2019)}}
\end{center}

\vspace{0.3in}
\noindent
\underline{Instructions}
\begin{itemize}
\vspace{0.1in}
\item You must print the problem set, write solutions on the printout (not on a separate sheet of paper), and staple all pages together.  Alternatively,  You will lose 1 point for not following this rule.
\vspace{0.1in}
\item Check the list of definitions and mathematical results you are allowed to use (available on Canvas).  You may not quote theorems found in a random book, including AAR.
\vspace{0.1in}
\item No late submission will be accepted.
\vspace{0.1in}
\item Each problem set will be graded out of 20 points based on your work on a few selected problems and the completeness of the rest of homework.
\vspace{0.1in}
\item You will receive 1 point for typing up your solutions using \LaTeX, as long as your problem set score does not exceed 20 points.  If you decide to do so, then download the corresponding tex file from Canvas and modify it (do not erase or change problem statements).  All VT students receive free Pro accounts on \href{https://www.overleaf.com/edu/vtech}{Overleaf}.
\vspace{0.1in}
\item You are encouraged to work with your classmates, but \textbf{you must write up your own solutions}.  Any form of plagiarism (e.g. copying your friend's tex file) will not be tolerated and will be reported to the Office of Academic Integrity.
\vspace{0.1in}
\item My solution will be posted within 24 hours of the deadline.  If there is a major mistake, the first student to point it out will receive 1 point.  I highly recommend that you make a copy of your problem set before you submit so that you can correct your mistakes immediately.
\end{itemize}

\vfill
\noindent
\textbf{Honor Pledge}: I did not give or receive any unauthorized assistance on this problem set.

\vspace{0.5in}
\hfill\makebox[3.0in]{\hrulefill}

\hfill\makebox[3.0in]{Signature}

\newpage
\begin{enumerate}[{\bfseries 1.}]
\item\begin{enumerate}[(a)]
	\vspace{0.1in}
	\item Below is a ``proof'' that $\sqrt{2}$ is rational.  What is wrong with it?

		\vspace{0.1in}
		{\begin{proof}
		Let $d_{n}$ be the $n$th decimal of $\sqrt{2}$ (e.g. $d_{1}=4$, $d_{2}=1$, $d_{3}=4$, $d_{4}=2$), and define $\displaystyle a_{n}=1+\sum_{i=1}^{n}\frac{d_{i}}{10^{i}}$ (e.g. $a_{1}=1.4$, $a_{2}=1.41$, $a_{3}=1.414$, $a_{4}=1.4142$).  We will prove by induction that, for every integer $n\geq 1$,  $P(n):\ ``a_{n}\text{ is rational}"$ is true.
		
		\vspace{0.1in}
		\underline{Base case}

		$a_{1}=1.4=\dfrac{7}{5}$ is rational, so $P(1)$ is true.

		\vspace{0.1in}
		\underline{Induction step}

		Let $k\geq 1$ be an integer, and suppose $P(k)$ is true.  Then, we have
		$$\displaystyle a_{k+1}=1+\sum_{i=1}^{k+1}\frac{d_{i}}{10^{i}}=\left(1+\sum_{i=1}^{k}\frac{d_{i}}{10^{i}}\right)+\frac{d_{k+1}}{10^{k+1}}=a_{k}+\frac{d_{k+1}}{10^{k+1}}$$
		is rational (note: $\mathbb{Q}$ is closed under addition), i.e. $P(k+1)$ is true.
		
		\vspace{0.1in}
		Therefore, $P(\infty)$ is true by mathematical induction.  Because $\displaystyle a_{\infty}=1+\sum_{i=1}^{\infty}\frac{d_{i}}{10^{i}}=\sqrt{2}$, this means that $\sqrt{2}$ is rational.
		\end{proof}}
		\\We cannot prove $P(\infty)$ because there is no way to reach infinity and infinity is not a number.
		\vfill
	\vspace{0.1in}
	\item Is it possible to prove statements of the form ``for every real number $x\geq x_{0}$, $P(x)$'' using modified mathematical induction?  For example, consider the statement ``for every positive real number $x\geq 1$, $2x$ is a real number.''  Can you use mathematical induction with the base case $P(1)$ and the induction step $P(x)\to P(x+0.001)$ to prove the statement?  Why or why not?
		\\You cannot use mathematical induction because you can never minimize the step you take in rational set, which mean each of the statement is not connected tightly one after another. We cannot find the smallest step, hence, cannot prove every statement in between.
		\vfill
	\end{enumerate}

\newpage
\item Consider the following statement.

	\vspace{0.1in}
	\begin{center}
	\noindent
	\textit{For every integer $n\geq 1$, $\displaystyle\sum_{i=1}^{n}\frac{1}{2^{i}}=1-\frac{1}{2^{n}}$.}
	\end{center}

	\begin{enumerate}[(a)]
	\vspace{0.1in}
	\item Identify $P(n)$ and $n_{0}$.
	\vspace{0.1in}
	\\$n_{0} = 1$
	\\$P(n) = \frac{1}{2^1} + \frac{1}{2^2} + ... + \frac{1}{2^n} = 1-\frac{1}{2^n}$

	\vspace{0.1in}
	\item What statement do we need to prove in the induction step?
    	\vspace{0.1in}
    	\\For every integer $k\geq 1$, if $\frac{1}{2^1} + \frac{1}{2^2} + ... + \frac{1}{2^k} = 1-\frac{1}{2^k}$, then $\frac{1}{2^1} + \frac{1}{2^2} + ... + \frac{1}{2^{k+1}} = 1-\frac{1}{2^{k+1}}$

	\vspace{0.1in}
	\item Use mathematical induction to prove the statement.
    	\begin{proof}
    	\vspace{0.1in}
    	We will prove by induction that, for every integer $n \geq 1$, $P(n)$: $\frac{1}{2^1} + \frac{1}{2^2} + ... + \frac{1}{2^n} = 1-\frac{1}{2^n}$ is true.
    	\\\underline{Base case}:
    	\\$\frac{1}{2^1} = \frac{1}{2} = 1 - \frac{1}{2^1}$, so $P(1)$ is true.
    	\\\underline{Induction step}:
    	\\Let $k\geq 1$ be an integer, suppose $P(k)$ is true.
    	\\Then we have
		\begin{align*}
		\frac{1}{2^1} + \frac{1}{2^2} + ... + \frac{1}{2^k} + \frac{1}{2^{k+1}}&= 1-\frac{1}{2^k} + \frac{1}{2^{k+1}} \text{ (by the induction hypothesis)} \\
		&= 1 - \frac{1}{2^k} + \frac{1}{2^k}\dotc \frac{1}{2} \\
		&= 1 - \frac{1}{2^k}(1 - \frac{1}{2}) \\
		&= 1 - \frac{1}{2^k}\dotc \frac{1}{2} \\ 
		&= 1 - \frac{1}{2^{k+1}}
		\end{align*}
		\end{proof}
	\end{enumerate}

\newpage
\item Consider the following statement.

	\vspace{0.1in}
	\begin{center}
	\noindent
	\textit{For every integer $n\geq 1$, $\displaystyle\sum_{i=1}^{n}i^{2}\leq n^{3}$.}
	\end{center}

	\begin{enumerate}[(a)]
	\vspace{0.1in}
	\item Identify $P(n)$ and $n_{0}$.
		\\$n_{0} = 1$
	    \\$P(n) = 1^2 + 2^2 + ... + n^2 \leq n^3$

	\vspace{0.1in}
	\item What statement do we need to prove in the induction step?
		\vspace{0.1in}
    	\\For every integer $k\geq 1$, if $1^2 + 2^2 + ... + n^2 \leq n^3$, then $1^2 + 2^2 + ... + (n+1)^2 \leq (n+1)^3$

	\vspace{0.1in}
	\item Use mathematical induction to prove the statement.
		\begin{proof}
    	\vspace{0.1in}
    	We will prove by induction that, for every integer $n \geq 1$, $P(n)$: $1^2 + 2^2 + ... + n^2 \leq n^3$ is true.
    	\\\underline{Base case}:
    	\\$1^2 = 1 \leq 1^3 = 1$, so $P(1)$ is true.
    	\\\underline{Induction step}:
    	\\Let $k\geq 1$ be an integer, suppose $P(k)$ is true.
    	\\Then we have
		\begin{align*}
		1^2 + 2^2 + ... + k^2 + (k+1)^2 &\leq k^3 + (k+1)^2 \text{ (by the induction hypothesis)} \\
		&\leq k^3 + k^2 + 2k + 1\\
		&\leq k^3 + 3k^2 + 3k + 1  \\
		&= (k+1)^3
		\end{align*}
		\end{proof}
	\end{enumerate}
\end{enumerate}

\newpage
\noindent
\underline{Definition}

\vspace{0.05in}
\noindent
The \textbf{Fibonacci sequence} is the sequence $\{f_{n}\}$ defined by $f_{1}=f_{2}=1$ and $f_{n}=f_{n-1}+f_{n-2}$ for every integer $n\geq 3$.

\vspace{0.3in}
\begin{enumerate}[{\bfseries 1.}]
\addtocounter{enumi}{3}
\item Consider the following statement about the Fibonacci sequence.

	\vspace{0.1in}
	\begin{center}
	\noindent
	\textit{For every integer $n\geq 1$, $\displaystyle\sum_{i=1}^{n}f_{2i-1}=f_{2n}$.}
	\end{center}

	\begin{enumerate}[(a)]
	\vspace{0.1in}
	\item Identify $P(n)$ and $n_{0}$.
		\\$n_{0} = 1$
	    \\$P(n) = f_{1} + f_{3} + ... + f_{2n - 1} = f_{2n}$

	\vspace{0.1in}
	\item What statement do we need to prove in the induction step?
		\vspace{0.1in}
    	\\For every integer $k\geq 1$, if $f_{1} + f_{3} + ... + f_{2n - 1} = f_{2n}$, then $f_{1} + f_{3} + ... + f_{2n + 1} = f_{2n+2}$

	\vspace{0.1in}
	\item Use mathematical induction to prove the statement.
		\begin{proof}
    	\vspace{0.1in}
    	We will prove by induction that, for every integer $n \geq 1$, $P(n)$: $f_{1} + f_{3} + ... + f_{2n - 1} = f_{2n}$ is true.
    	\\\underline{Base case}:
    	\\$f_1 = 1 = f_2$, so $P(1)$ is true.
    	\\\underline{Induction step}:
    	\\Let $k\geq 1$ be an integer, suppose $P(k)$ is true.
    	\\Then we have
		\begin{align*}
	    f_{1} + f_{3} + ... + f_{2k - 1} + f_{2k+1} &=  f_{2k} + f_{2k+1}\text{ (by the induction hypothesis)} \\
		&= f_{2k+2} \text{ (by definition of Fibonacci sequence)}\\
		\end{align*}
		\end{proof}
	\end{enumerate}

\newpage
\item Recall that \footnotemark[2]two non-parallel lines intersect at one point on a plane.  Let $a_{n}$ be the number of intersections produced by $n$ non-parallel lines (no more than two lines may intersect at the same point).

	\begin{enumerate}[(a)]
	\vspace{0.1in}
	\item Determine $a_{2}$, $a_{3}$, $a_{4}$, and $a_{5}$.
		\\$a_{2} = 1$
		\\$a_{3} = 3$
		\\$a_{4} = 6$
		\\$a_{5} = 10$

	\vspace{0.1in}
	\item \footnotemark[3]Formulate a conjecture about $a_{n}$ (i.e. express $a_{n}$ in terms of $n$).
		\\For every integer $n\geq 3$, $a_{n} = \frac{(n-1)(n)}{2} =  \displaystyle\sum_{i=1}^{n-1}i$

	\vspace{0.1in}
	\item Use mathematical induction to prove your conjecture.
		\begin{proof}
    	\vspace{0.1in}
    	We will prove by induction that, for every integer $n \geq 3$, $P(n)$: $a_{n} =  \displaystyle\sum_{i=1}^{n-1}i = 1 + 2 + ... + (n-1) = \frac{n(n-1)}{2}$ is true.
    	\\\underline{Base case}:
    	\\$a_{3} = 1 + 2 = 3$, so $P(3)$ is true.
    	\\\underline{Induction step}:
    	\\Let $k\geq 3$ be an integer, suppose $P(k)$ are true.
    	\\Then we have
		\begin{align*}
	    1 + 2 + ... + k-1 + k &= \frac{k(k-1)}{2} + k\text{ (by the induction hypothesis)} \\
		&= \frac{k(k-1)+2k}{2} \\
		&= \frac{k(k-1+2)}{2}\\
		&= \frac{k(k+1)}{2} \\
		&= a_{k+1}
		\end{align*}
		\end{proof}
	\end{enumerate}
	\footnotetext[2]{This is not true on other surfaces.  For example, consider \href{https://en.wikipedia.org/wiki/Spherical_geometry}{spherical geometry}.}
	\footnotetext[3]{You should figure this out on your own, but if it is too hard, then try \href{https://oeis.org/}{OEIS}.}
\end{enumerate}
\end{document}
