\documentclass[12pt]{amsart}
\usepackage{amsmath, amsthm, graphicx, fullpage, bm, mathrsfs, mathabx, enumerate, marginnote, color, hyperref, wasysym}

%%%%%for footnote%%%%%
\usepackage[symbol]{footmisc}
\renewcommand{\thefootnote}{\fnsymbol{footnote}}
%%%%%%%%%%%%%%%%

%%%%%set of commands defined by me%%%%%
\newcommand{\abs}[1]{\ensuremath{\left\lvert #1\right\rvert}}
\newcommand*\diff{\mathop{}\!\mathrm{d}}
\newcommand{\NOT}[1]{\ensuremath{\sim\hspace{-0.05in}#1}}
\renewcommand*\divides{\hspace{0.05in}|\hspace{0.05in}}
\renewcommand*\div{\hspace{0.05in}\text{div}\hspace{0.05in}}
\renewcommand*\mod{\hspace{0.05in}\text{mod}\hspace{0.05in}}
\newcommand{\floor}[1]{\ensuremath{\left\lfloor #1\right\rfloor}}
\newcommand{\ceil}[1]{\ensuremath{\left\lceil #1\right\rceil}}
%%%%%%%%%%%%%%%%%%%%%%%%%%

%%%%%%%%%%%%%%%%%%%%%%%%%%

%%%%%%%%%%%%%%%%%%set notation%%%%%%%%%%%%%%%%%%
%https://tex.stackexchange.com/questions/253077/how-do-you-create-a-set-in-latex
\usepackage{mathtools}
\DeclarePairedDelimiterX\set[1]\lbrace\rbrace{\def\given{\;\delimsize\vert\;}#1}
%%%%%%%%%%%%%%%%%%%%%%%%%%%%%%%%%%%%%%%%%%

\begin{document}
%\pagenumbering{gobble}

\hfill\makebox[2.5in][l]{\textbf{Name: Skylar Liang}}

\vspace{0.3in}
\begin{center}
{\Large{\textbf{MATH 3034}}}

\vspace{0.1in}
{\large{Problem Set 6 (Due Wednesday, October 30, 2019)}}
\end{center}

\vspace{0.3in}
\noindent
\underline{Instructions}
\begin{itemize}
\vspace{0.1in}
\item You must print the problem set, write solutions on the printout (not on a separate sheet of paper), and staple all pages together.  Alternatively,  You will lose 1 point for not following this rule.
\vspace{0.1in}
\item Check the list of definitions and mathematical results you are allowed to use (available on Canvas).  You may not quote theorems found in a random book, including AAR.
\vspace{0.1in}
\item No late submission will be accepted.
\vspace{0.1in}
\item Each problem set will be graded out of 20 points based on your work on a few selected problems and the completeness of the rest of homework.
\vspace{0.1in}
\item You will receive 1 point for typing up your solutions using \LaTeX, as long as your problem set score does not exceed 20 points.  If you decide to do so, then download the corresponding tex file from Canvas and modify it (do not erase or change problem statements).  All VT students receive free Pro accounts on \href{https://www.overleaf.com/edu/vtech}{Overleaf}.
\vspace{0.1in}
\item You are encouraged to work with your classmates, but \textbf{you must write up your own solutions}.  Any form of plagiarism (e.g. copying your friend's tex file) will not be tolerated and will be reported to the Office of Academic Integrity.
\vspace{0.1in}
\item My solution will be posted within 24 hours of the deadline.  If there is a major mistake, the first student to point it out will receive 1 point.  I highly recommend that you make a copy of your problem set before you submit so that you can correct your mistakes immediately.
\end{itemize}

\vfill
\noindent
\textbf{Honor Pledge}: I did not give or receive any unauthorized assistance on this problem set.

\vspace{0.5in}
\hfill\makebox[3.0in]{\hrulefill}

\hfill\makebox[3.0in]{Signature}

\newpage
\begin{enumerate}[{\bfseries 1.}]
\item Use strong mathematical induction to prove the following statement about the Fibonacci sequence.

	\vspace{0.1in}
	\begin{center}
	\noindent
	\textit{For every integer $n\geq 1$, $f_{n}=\dfrac{1}{\sqrt{5}}\left[\left(\dfrac{1+\sqrt{5}}{2}\right)^{n}-\left(\dfrac{1-\sqrt{5}}{2}\right)^{n}\right]$.}
	\end{center}
	\begin{proof}
	\\We will prove by induction that, for every integer $n \geq 1$, 
	\\$P(n):$ $f_{n}=\dfrac{1}{\sqrt{5}}\left[\left(\dfrac{1+\sqrt{5}}{2}\right)^{n}-\left(\dfrac{1-\sqrt{5}}{2}\right)^{n}\right]$
	\\\underline{Base cases}
	\\$f(1) =\dfrac{1}{\sqrt{5}}\left[\left(\dfrac{1+\sqrt{5}}{2}\right)^{1}-\left(\dfrac{1-\sqrt{5}}{2}\right)^{1}\right] = 1$
	\\$f(2) = \dfrac{1}{\sqrt{5}}\left[\left(\dfrac{1+\sqrt{5}}{2}\right)^{2}-\left(\dfrac{1-\sqrt{5}}{2}\right)^{2}\right] = 2$
	\\so $P(1)$ and $P(2)$ are true.
	\\\underline{Induction step}
	\\Let $k\geq 2$ be an integer and suppose $P(k-1), P(k)$ are true.
	\\Because $f_{k+1}$ is a Fibonacci sequence, $f_{k+1} = f_{k} + f_{k-1}$.
	\begin{align*}
	    f_{k+1} &= f_{k} + f_{k-1}\\
		&= \dfrac{1}{\sqrt{5}}\left[\left(\dfrac{1+\sqrt{5}}{2}\right)^{k}-\left(\dfrac{1-\sqrt{5}}{2}\right)^{k}\right] + \dfrac{1}{\sqrt{5}}\left[\left(\dfrac{1+\sqrt{5}}{2}\right)^{k-1}-\left(\dfrac{1-\sqrt{5}}{2}\right)^{k-1}\right]
		\\\text{(by the induction hypothesis)} \\
		&= \dfrac{1}{\sqrt{5}}\left[\left(\dfrac{1+\sqrt{5}}{2}\right)^{k}-\left(\dfrac{1-\sqrt{5}}{2}\right)^{k}+\left(\dfrac{1+\sqrt{5}}{2}\right)^{k-1}-\left(\dfrac{1-\sqrt{5}}{2}\right)^{k-1}\right]\\
		&= \dfrac{1}{\sqrt{5}}\left[\left(\dfrac{1+\sqrt{5}}{2}\right)^{k}-\left(\dfrac{1-\sqrt{5}}{2}\right)^{k}+\left(\dfrac{1+\sqrt{5}}{2}\right)^{k-1}-\left(\dfrac{1-\sqrt{5}}{2}\right)^{k-1}\right]
		\end{align*}
	\end{proof}
\newpage
\item Consider the sequence $\{a_{n}\}$ defined by $a_{1}=3$, $a_{2}=7$, and $a_{n}=3a_{n-1}-2a_{n-2}$ for every integer $n\geq 3$.

	\begin{enumerate}[(a)]
	\vspace{0.1in}
	\item Compute $a_{3}$, $a_{4}$, and $a_{5}$.
		\\$a_{3} = 15$
		\\$a_{4} = 31$
		\\$a_{5} = 63$
	\vspace{0.1in}
	\item Formulate a conjecture about $a_{n}$ (i.e. express $a_{n}$ in terms of $n$).
		\\$a_{1} = 3, a_{2} = 5,$ and $a_{n} = 3a_{n-1}-2a_{n-2}.$ For every integer $n\geq 1, a_{n} = 2^{n+1} - 1$
	\vspace{0.1in}
	\item Use strong mathematical induction to prove your conjecture.
		\begin{proof}
	    \\We will prove by induction that, for every integer $n\geq 1$, $P(n)$: $a_n = 2^{n+1} -1$ is true.
		\\\underline{Base cases}
		\\$a_{1} = 2^{1+1} - 1 = 3$, and $a_{2} = 2^{2+1} - 1 = 7$, so $P(1)$ and $P(2)$ are true.
		\\\underline{Induction step}
		\\Let $k\geq 2$ be an integer and suppose $P(k-1)$ and $P(k)$ are true. 
		\begin{align*}
	    a_{k+1} &= 3a_k - 2a_{k-1} \\
		        &= 3(2^{k+1} - 1) - 2(2^{k-1+1} -1) \text{ (by the induction hypothesis)} \\
		        &= 3\cdot 2^{k+1}-3-2\cdot 2^k +2\\
		        &= 3\cdot 2^{k+1}-2^{k+1}-1\\
		        &= 2\cdot 2^{k+1} -1 \\
		        &= 2^{k+1+1}-1
		\end{align*}
		\end{proof}
		\vfill
		\vfill
		\vfill
	\end{enumerate}

\newpage
\item Consider the following statement.

	\vspace{0.1in}
	\begin{center}
	\noindent
	\textit{For every integer $n\geq 8$, there exist \textbf{nonnegative} integers $a$ and $b$ such that $n=3a+5b$.}
	\end{center}
	
	\vspace{0.1in}
	\noindent
	(i.e. every integer that is greater than 8 can be expressed as a sum of 3's and 5's.)

	\begin{enumerate}[(a)]
	\vspace{0.1in}
	\item Verify the statement for $8\leq n\leq 13$.  Do you notice any pattern?
		\\$n = 8 = 3\cdot 1 + 5\cdot 1$
		\\$n = 9 = 3\cdot 3 + 5\cdot 0$
		\\$n = 10 = 3\cdot 0 + 5\cdot 2$
		\\$n = 11 = 3\cdot 2 + 5\cdot 1$
		\\$n = 12 = 3\cdot 4 + 5\cdot 0$
		\\$n = 13 = 3\cdot 1 + 5\cdot 2$
	\vspace{0.1in}
	\item Use strong mathematical induction to prove the statement.
		\begin{proof}
		\\We will prove by induction that, for every integer $n\geq 8$, $P(n)$: $n = 3a+5b$ where $a, b\in\mathbb{Z}$ and $a\geq0, b\geq0$ is true.
		\\\underline{Base cases}
		\\$8 = 3\cdot 1 + 5\cdot 1$, where $a = 1, b = 1$, so $P(8)$ is true.
		\\\underline{Induction step}
		\\Let $k\geq 8$ be an integer and suppose $P(8),...,P(k)$ are true. Because $P(k-2)$ is true, there exist $m,n\in\mathbb{Z}, m\geq 0, n\geq0$ such that $k-2=3m+5n$.
		\begin{align*}
	    k+1 &= (k-2) + 3\\
		&= 3m+5n+3 \text{ (by the induction hypothesis)} \\
		&= 3(m+1) + 5n
		\end{align*}
		where $m+1\geq 0, n\geq 0$.
		\end{proof}
		\vfill
		\vfill
	\end{enumerate}

\newpage
\item For the following pairs $(n, d)$ of integers, find the integers $q$ and $r$ given by the quotient-remainder theorem.

	\begin{enumerate}[(a)]
	\vspace{0.1in}
	\item $(n, d)=(50, 8)$
		\\$q = 6, n = 2$
	\vspace{0.1in}
	\item $(n, d)=(-20, 7)$
		\\$q = -3, r = 1$
	\end{enumerate}

\vspace{0.2in}
\item Let $n\in\mathbb{Z}$.  Prove the following statement.  You may use any proof technique.

	\vspace{0.1in}
	\begin{center}
	\noindent
	\textit{\footnotemark[2]If $3\divides n^{2}$, then $3\divides n$.}
	\end{center}

	 \begin{proof}
	 \\We will prove the contrapositive.
	 \\Suppose $3$ doesn't divide $n$, then by Quotient-Remainder Theorem, there exits $k\in\mathbb{Z}$ such that $n = 3k+1$ or $n = 3k +2$.
	 \\\underline{Case 1: ($n = 3k+1$)}
	 \\Because $n = 3k +1$, we have $n^2 = 9k^2 + 6k +1 = 3(3k^2 + 3k) + 1$. Therefore 3 doesn't divide $n^2$.
	 \\\underline{Case 2: ($n = 3k+2$)}
	 \\Because $n = 3k +2$, we have $n^2 = 9k^2 + 12k +4 = 3(3k^2 + 4k+1) + 1$. Therefore 3 doesn't divide $n^2$.
	 
	 \end{proof}
	 \vfill
	 \vfill
	 \vfill
	 \vfill

	\footnotetext[2]{Recall that you used this result to prove the irrationality of $\sqrt{3}$ (Problem Set 3).}

\newpage
\item\begin{enumerate}[(a)]
	\vspace{0.1in}
	\item Use the Euclidean algorithm to find $\gcd{(536, 100)}$.
		\\$536 = 100(5) + 36$
		\\$100 = 36(2) + 28$
		\\$36 = 28(1) + 8$
		\\$28 = 8(3)+4$
		\\$8 = 4(2) + 0$
		\\$\gcd{(536, 100)} = \gcd{(4, 0)} = 4$
	\vspace{0.1in}
	\item Find integers $m$ and $n$ such that $\gcd{(536, 100)}=536m+100n$.  Show work.
		\\$36 = 536 - 100(5)$
		\\$28 = 100 - 36(2)$
		\\$8 = 36 - 28(1)$
		\\$4 = 28 - 8(3)$
		\begin{align*}
		4 &= 28 - 3(36 - 28) \\
		&= 4\cdt 28 - 3\cdot 36 \\
		&= (4)100 - (8)36-(3)536+(15)100 \\
		&= (59)100 - (11)536
		\end{align*}
	\end{enumerate}

\end{enumerate}
\end{document}
