\documentclass[12pt]{amsart}
\usepackage{amsmath, amsthm, graphicx, fullpage, bm, mathrsfs, mathabx, enumerate, marginnote, color, hyperref, wasysym}

%%%%%for footnote%%%%%
\usepackage[symbol]{footmisc}
\renewcommand{\thefootnote}{\fnsymbol{footnote}}
%%%%%%%%%%%%%%%%

%%%%%set of commands defined by me%%%%%
\newcommand{\abs}[1]{\ensuremath{\left\lvert #1\right\rvert}}
\newcommand*\diff{\mathop{}\!\mathrm{d}}
\newcommand{\NOT}[1]{\ensuremath{\sim\hspace{-0.05in}#1}}
\renewcommand*\divides{\hspace{0.05in}|\hspace{0.05in}}
\renewcommand*\div{\hspace{0.05in}\text{div}\hspace{0.05in}}
\renewcommand*\mod{\hspace{0.05in}\text{mod}\hspace{0.05in}}
\newcommand{\floor}[1]{\ensuremath{\left\lfloor #1\right\rfloor}}
\newcommand{\ceil}[1]{\ensuremath{\left\lceil #1\right\rceil}}
%%%%%%%%%%%%%%%%%%%%%%%%%%

%%%%%%%%%%%%%%%%%%%%%%%%%%

%%%%%%%%%%%%%%%%%%set notation%%%%%%%%%%%%%%%%%%
%https://tex.stackexchange.com/questions/253077/how-do-you-create-a-set-in-latex
\usepackage{mathtools}
\DeclarePairedDelimiterX\set[1]\lbrace\rbrace{\def\given{\;\delimsize\vert\;}#1}
%%%%%%%%%%%%%%%%%%%%%%%%%%%%%%%%%%%%%%%%%%

\begin{document}
%\pagenumbering{gobble}

\hfill\makebox[2.5in][l]{\textbf{Name: Skylar Liang}}

\vspace{0.3in}
\begin{center}
{\Large{\textbf{MATH 3034}}}

\vspace{0.1in}
{\large{Problem Set 7 (Due Wednesday, November 20, 2019)}}
\end{center}

\vspace{0.3in}
\noindent
\underline{Instructions}
\begin{itemize}
\vspace{0.1in}
\item You must print the problem set, write solutions on the printout (not on a separate sheet of paper), and staple all pages together.  Alternatively,  You will lose 1 point for not following this rule.
\vspace{0.1in}
\item Check the list of definitions and mathematical results you are allowed to use (available on Canvas).  You may not quote theorems found in a random book, including AAR.
\vspace{0.1in}
\item No late submission will be accepted.
\vspace{0.1in}
\item Each problem set will be graded out of 20 points based on your work on a few selected problems and the completeness of the rest of homework.
\vspace{0.1in}
\item You will receive 1 point for typing up your solutions using \LaTeX, as long as your problem set score does not exceed 20 points.  If you decide to do so, then download the corresponding tex file from Canvas and modify it (do not erase or change problem statements).  All VT students receive free Pro accounts on \href{https://www.overleaf.com/edu/vtech}{Overleaf}.
\vspace{0.1in}
\item You are encouraged to work with your classmates, but \textbf{you must write up your own solutions}.  Any form of plagiarism (e.g. copying your friend's tex file) will not be tolerated and will be reported to the Office of Academic Integrity.
\vspace{0.1in}
\item My solution will be posted within 24 hours of the deadline.  If there is a major mistake, the first student to point it out will receive 1 point.  I highly recommend that you make a copy of your problem set before you submit so that you can correct your mistakes immediately.
\end{itemize}

\vfill
\noindent
\textbf{Honor Pledge}: I did not give or receive any unauthorized assistance on this problem set.

\vspace{0.5in}
\hfill\makebox[3.0in]{\hrulefill}

\hfill\makebox[3.0in]{Signature}

\newpage
\begin{enumerate}[{\bfseries 1.}]
\item Consider the binary relation $\sim$ on $\mathbb{R}$ defined as follows:

	$$\footnotemark[2] x\sim y\Longleftrightarrow \abs{x-y}<0.1.$$

	\begin{enumerate}[(a)]
	\vspace{0.2in}
	\item Determine if $\sim$ is reflexive.  Prove your claim.
	    \\$\sim$ is reflexive.
		\begin{proof}
		Let $x \in\mathbb{R}$, then $\abs{x - x} = 0 < 0.1$, so $x\sim x$.
		\end{proof}
	\vspace{0.1in}
	\item Determine if $\sim$ is symmetric.  Prove your claim.
	    \\$\sim$ is symmetric.
		\begin{proof}
		Let $x, y \in\mathbb{R}$ such that $x\sim y$, then $\abs{x - y} < 0.1$, so $-0.1 < (x - y) < 0.1 \iff 0.1 > y - x > -0.1 \iff \abs{y - x} < 0.1$, so $y\sim x$.
		\end{proof}
	\vspace{0.1in}
	\item Determine if $\sim$ is transitive.  Prove your claim.
	    \\$\sim$ is not transitive.
		\begin{proof}
		Suppose $\abs{2.08 - 2} = 0.08 < 0.1$, and $\abs{2 - 1.92} = 0.08 < 0.1$, but we have $\abs{2.08 - 1.92} = 1.6 > 0.1$.
		\end{proof}
	\end{enumerate}
	\footnotetext[2]{Note that this idea may be used to define approximate equality (e.g. $0.99\approx 1$ and $\pi\approx 3.14$, but $1\not\approx 2$).}

\newpage
\item\begin{enumerate}[(a)]
	\vspace{0.1in}
	\item Construct a binary relation $\sim$ on $\mathbb{R}$ that is symmetric, but not reflexive and not transitive.

		\vspace{0.3in}
		$x\sim y\Longleftrightarrow x = \frac{1}{y}$

	\vspace{0.3in}
	\item Prove that $\sim$ is not reflexive.
		\begin{proof}
		Because $2 \neq \frac{1}{2}$, $x \nsim x$.
		\end{proof}
	\vspace{0.1in}
	\item Prove that $\sim$ is symmetric.
		\begin{proof}
		Let $x, y\in\mathbb{R}$ such that $x\sim y$, then $x = \frac{1}{y}$, and $xy = 1 \iff y = \frac{1}{x}$, so $y\sim x$.
		\end{proof}
	\vspace{0.1in}
	\item Prove that $\sim$ is not transitive.
		\begin{proof}
		Because $2 = \frac{1}{\frac{1}{2}}$, and $\frac{1}{2} = \frac{1}{2}$, so $2 \sim \frac{1}{2}$, and $\frac{1}{2} \sim 2$, but $2 \neq \frac{1}{2}$, so $2\nsim 2$.
		\end{proof}
	\end{enumerate}

\newpage
\item\begin{enumerate}[(a)]
	\vspace{0.1in}
	\item Construct a binary relation $\sim$ on $\mathbb{R}$ that is transitive, but not reflexive and not symmetric.

		\vspace{0.3in}
		$x\sim y\Longleftrightarrow y = x^{\lceil x \rceil n}$ for some n $\in\mathbb{Z}$

	\vspace{0.3in}
	\item Prove that $\sim$ is not reflexive.
		\begin{proof}
		Because $0^0 = 1 \neq 0$, $0\nsim 0$.
		\end{proof}
	\vspace{0.1in}
	\item Prove that $\sim$ is not symmetric.
		\begin{proof}
		Because $4 = 2^2$ where $n = 1$, and $2 = 4^{\frac{1}{2}}$ where $n = \frac{1}{8}$, we have $2\sim 4$ but $4\nsim 2$.
		\end{proof}
	\vspace{0.1in}
	\item Prove that $\sim$ is transitive.
	    \begin{proof}
		Let $x, y, z\in\mathbb{R}$ such that $x\sim y$ and $y\sim z$.
		\\Because $x\sim y$, there exists $m\in\mathbb{Z}$ such that $y = x^{\lceil x \rceil m}$.
		\\Similarly, because $y\sim z$, there exists $n\in\mathbb{Z}$ such that $z = y^{\lceil x \rceil n}$.
		\\Now we have $z = (x^{\lceil x \rceil m})^{\lceil x \rceil n} = (x ^ {\lceil x \rceil})^{\lceil x \rceil mn}$ where by definition $\lceil x \rceil \in\mathbb{Z}$ and $mn\in\mathbb{Z}$, so $x\sim z$.
		\end{proof}
	\end{enumerate}
\end{enumerate}

\newpage
\noindent
\underline{Recall}

\vspace{0.05in}
\noindent
Let $m, n\in\mathbb{Z}^{+}$.  The set of all $m\times n$ matrices with real entries is denoted by $M_{m,n}(\mathbb{R})$. \\
If $m=n$, then we also write $M_{n}(\mathbb{R})$.

\vspace{0.3in}
\begin{enumerate}[{\bfseries 1.}]
\addtocounter{enumi}{3}
\item Consider the equivalence relation on $M_{2}(\mathbb{R})$ defined as follows:

	$$A\sim B\Longleftrightarrow \det{A}=\det{B}.$$

	\begin{enumerate}[(a)]
	\item\begin{enumerate}[(i)]
		\vspace{0.2in}
		\item Give a set description of $[I_{2}]$.
			\\$[I_{2}] =  \left\{A \in M_{2}(\mathbb{R})|A \sim I_{2}\right\}
			\\=  \left\{A \in M_{2}(\mathbb{R})|\det{A}=\det{I_{2}}\right\}
			\\=  \left\{A \in M_{2}(\mathbb{R})|\det{A}=1 \right\}$
		\vspace{0.1in}
		\item List 3 distinct representatives for $[I_{2}]$.
		\\
			 $\begin{bmatrix}
    			1 & 2 \\
    			0 & 1 
			 \end{bmatrix}, 
			 \begin{bmatrix}
    			1 & 3 \\
    			0 & 1 
			 \end{bmatrix},
			 \begin{bmatrix}
    			1 & 1 \\
    			0 & 1 
			 \end{bmatrix}$,
		\end{enumerate}
	\vspace{0.1in}
	\item Determine all distinct equivalence classes (without double counting).
		\\$\begin{bmatrix}
    		a & b \\
    		c & d 
    	 \end{bmatrix}$ where $a,b,c,d\in\mathbb{R}$ and $ad - bc = 1$
	\end{enumerate}

\vspace{0.2in}
\item Consider the equivalence relation on $\mathbb{Z}^{+}\times\mathbb{Z}^{+}$ defined as follows:

	$$(x_{1}, x_{2})\sim (y_{1}, y_{2})\Longleftrightarrow x_{1}y_{2}=x_{2}y_{1}.$$

	\begin{enumerate}[(a)]
	\item\begin{enumerate}[(i)]
		\vspace{0.2in}
		\item Give a set description of $[(2,4)]$.
			\\$[(2,4)] =  \left\{((x_{1}, x_{2}) \in\mathbb{Z^{+}\times Z^{+}}| 4x_{1} = 2x_{2}\right\}$
		\vspace{0.1in}
		\item List 3 distinct representatives for $[(2,4)]$.
			\\$(2, 4)$, $(1, 2)$, $(3, 6)$
		\end{enumerate}
	\vspace{0.1in}
	\item Determine all distinct equivalence classes (without double counting).
		\\$[(a, 2a)]$ where $a\in\mathbb{Z^+}$
	\end{enumerate}

\newpage
\item Determine if the given $f$ is a function.  If so, then what are the domain, codomain, and range of $f$?

	\begin{enumerate}[(a)]
	\vspace{0.1in}
	\item $f: \mathbb{R}\to\mathbb{Q}$ defined by $f(x)=\dfrac{3}{x^{2}+1}$
		\\domain: $\mathbb{R}$
		\\codomain: $\mathbb{Q}$
		\\range: $(0, 3]$
	\vspace{0.1in}
	\item $f: \mathbb{R}^{+}\to\mathbb{R}$ defined by $f(x)=x-\ceil{x}$
		\\domain: $\mathbb{R^+}$
		\\codomain: $\mathbb{R}$
		\\range: $(-1, 0]$
	\vspace{0.1in}
	\item $f: \mathbb{Z}^{+}\times\mathbb{Z}^{+}\to\mathbb{Q}$ defined by $f(m,n)=m-n$
		\\domain: $\mathbb{Z^+ \times Z^+}$
		\\codomain: $\mathbb{Q}$
		\\range: $\mathbb{Z^+} \cup 0$
	\vspace{0.1in}
	\item $f: P(\set{0, \pm 1, \pm 2})\to\mathbb{Z}$ defined by $f(S)=(\text{number of elements in }S)$
		\\domain: $\set{0, \pm 1, \pm 2}$
		\\codomain: $\mathbb{Z}$
		\\range: $15$
	\end{enumerate}

\newpage
\noindent
\item Consider the following functions.
	\begin{itemize}
	\vspace{0.2in}
	\item $f: (-\infty, -1]\to [0,\infty)$ defined by $f(x)=\sqrt{x^{2}-1}$
	\vspace{0.1in}
	\item $g: (\mathbb{R}-\mathbb{Z})\to\mathbb{R}$ defined by $g(x)=\tan{\left(\pi x+\dfrac{\pi}{2}\right)}$
	\vspace{0.1in}
	\item $h: \mathbb{Q}\to\mathbb{Q}$ defined by $h(x)=x^{3}$
	\end{itemize}

	\begin{enumerate}[(a)]
	\vspace{0.2in}
	\item\begin{enumerate}[(i)]
		\item Which function is 1-1 but not onto?
			\\$h$ is 1-1 but not onto.
		\vspace{0.1in}
		\item Prove that the function is not onto.
			\begin{proof}
			There does not exist $x\in\mathbb{Q}$ such that $x^3 = 2$, so $h$ is not onto.
			\end{proof}
		\end{enumerate}
	\vspace{0.1in}
	\item\begin{enumerate}[(i)]
		\item Which function is onto but not 1-1?
			\\$g$ is onto but not 1-1.
		\vspace{0.1in}
		\item Prove that the function is not 1-1.
			\begin{proof}
			Because $g(0.5) = 0 = g(1.5)$, so $g$ is not 1-1.
			\end{proof}
		\end{enumerate}
	\newpage
	\item\begin{enumerate}[(i)]
		\item Which function is a bijection?
			\\$f$ is a bijection.
		\vspace{0.1in}
		\item Prove that the function is 1-1.
			\begin{proof}
			Let $x_1, x_2\in (-\infty, -1]$ such that $f(x_1) = f(x_2)$, then we have $\sqrt{x_{1}^2 -1} = \sqrt{x_{2}^2 -1}$, square both sides, we have $x_{1}^2 = x_{2}^2 \Rightarrow x_1 = x_2$.
			\end{proof}
		\vspace{0.1in}
		\item Prove that the function is onto.
			\begin{proof}
			Let $y\in[0, \infty)$m let $x = -\sqrt{y^2 + 1}$, then $x\in(-\infty, -1]$ and $f(x) = f(-\sqrt{y^2 + 1}) = \sqrt{y^2 + 1 - 1} = \sqrt{y^2} = y$
			\end{proof}
		\end{enumerate}
	\end{enumerate}
\end{enumerate}
\end{document}
